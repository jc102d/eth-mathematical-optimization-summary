\documentclass[main]{subfiles}
\begin{document}

%@@@@@@@@@@@@@@@@@@@@@@@@@@@@@@
% Main Topics: 
% Algorithms for hard problems: knapsack problem - 20.11.2017
% author: Vanessa Leite

\section{Algorithms for hard problems}

\subsection{knapsack problem}
\paragraph{Definition - knapsack problem}

Given a set $C = \{ c_{1}, \dots, c_{n} \} \subseteq \mathbb{Z}_{+} \setminus
\{ 0\}$ (values) and $A = \{a_{1}, \dots, a_{n}\} \subseteq \mathbb{Z}_{+}
\setminus \{ 0\}$ (weights), $a_{0} \in \mathbb{Z}_{+} \setminus \{ 0\},
a_{0} \geq a_{i} \forall i$.

$\gamma(k, a_{0}) = \max \{\sum_{i =1}^{k} c_{i} x_{i} \mid \sum_{i=1}^{k}
a_{i}x_{i} \leq a_{0}, x \in \{0,1\}^{k} \}$ is a (binary) knapsack problem.
(If $x \in \mathbb{Z}_{+}^{k}$, then it is a integer knapsack problem).

Dual:
$\alpha(k, c_{0}) = \min \{ \sum_{i =1}^{k} a_{i} x_{i} \mid \sum_{i=1}^{k}
c_{i}x_{i} \geq c_{0}, x \in \{0,1\}^{k} \}$

\paragraph{A k-cursive algorithm}
For all $a \in \{1, \dots, \sum_{i=1}^{n} a_i\}$, initialize $\gamma(1, a_0)$:
\[
  \gamma(1, a_{0})=\begin{cases}
               0, a_{1} > a_{0}\\
               c_{1}, a_{1} \leq a_{0}\\
            \end{cases}
\]
For $k = 2, \dots, n$ and for $a = 1, \dots, \sum_{i=1}^{n} a_i$:
$\gamma(k, a) = \max \{\gamma(k-1, a), \gamma(k-1, a - a_k) + c_k \}$

The number of iterations for a given $a_{0}$ is $\mathcal{O}(a_{max} n^2)$.
Since $a_0 \leq n a_{max}$, $\mathcal{O}(n \times a_{0})$.

For the dual knapsack problem: for all $c \in \{1, \dots, \sum_{i=1}^{n} c_i
\}$, initialize $\alpha(1, c_0)$:
\[
  \alpha(1, c_{0})=\begin{cases}
               +\infty, c_{1} < c_{0}\\
               a_{1} \text{, if } 1 \leq c_0 \leq c_1\\
               0 \text{, if } c_0 = 0\\
            \end{cases}
\]
For $k = 2, \dots, n$ and for $c = 1, \dots, \sum_{i=1}^{n} c_i$:
$\alpha(k, c) = \min \{\alpha(k-1, c), \alpha(k-1, c - c_{k}) + a_{k}\}$

\paragraph{Corollary} The knapsack problem can be solved in time
$\mathcal{O}(n \times a_{0})$ and the dual version in time
$\mathcal{O}(n \times c_{0}) = \mathcal{O}(n^{2} \times max\{c_{i}: i =
\{1, \dots, n\}\})$

\paragraph{Notation} For numbers $d_{1}, \dots, d_{k}$, $d_{max} = max\{d_{i}:
i =\{1, \dots, k\} \}$

\subsection{Connections primal/dual knapsack}

\paragraph{Lemma} $\alpha(k, \gamma(k, a_{0}) \leq a_{0}$ and $\gamma(k,
\alpha(k, c_{0}) \geq c_{0}$

\subparagraph{Proof:}
Let $x^{*} \in \{0,1\}^{k}$ attain optimal value for $\gamma(k, a_{0})$.
Then:
\begin{itemize}
\item (i) $\sum a_{i}x_{i}^{*} \leq a_{0}$
\item (ii) $\sum c_{i}x_{i}^{*} = \gamma(k, a_{0})$
\end{itemize}

Consider the dual: $\min \{\sum_{i=1}^{k} a_i x_{i} \mid \sum_{i=1}^{k}
c_{i} x_{i} \geq \gamma(k, a_{0})\}$, $x^{*}$ is feasible for the dual,
$\leq \sum a_{i}x_{i}^{*} \leq a_{0}$.

\paragraph{Lemma} Let $\mathbb{A}$ be an algorithm for computing
$\alpha(k, c_{0}) \forall k$ and $c_{0}$. With "polynomial" many calls of
$\mathbb{A}$ we can compute $\gamma(k, a_{0})$.

\subparagraph{Proof:}
$\gamma(k, a_{0}) = \max c^{T}x, a^{T}x \leq a_{0}, x \in \{0,1\}^{k}$.
Assume $a_{0} \geq a_{max}$:

$\underbrace{0}_{\underline{\mu}} \leq \gamma(k, a_{0}) \leq
\underbrace{\sum_{i=1}^{k}c_{i}}_{\bar{\mu}}$.

Consider $\mu = \frac{\bar{\mu} + \underline{\mu}}{2}$. Compute
$\alpha(k, \mu)$, then two things ca happen:
\begin{itemize}
\item (i) the result is greather than $a_{0} \rightarrow \bar{\mu} = \mu$
\item (ii) the result is less or equal to $a_{0} \rightarrow \underline{\mu} =
\mu$

The number of steps that takes until $\underline{\mu} = \bar{\mu}$ is $log(k
\times c_{max})$.
\end{itemize}

\paragraph{Remark}
$\alpha(k, \mu)$ can be computed by finding a shortest path in a digraph
$D=(V,A)$, $V =\{0, \dots, n\} \times \{0, \dots, \sum_1^k c_i\}$. The arc $A$
is of the type $((i, \sigma^\prime), (i+1, \sigma)) \iff \sigma - \sigma^\prime
\in \{0, c_i + 1\}$ of weight $0$ (if $\sigma - \sigma^\prime = 0$) or
$a_i + 1$ (if $\sigma - \sigma^\prime = c_i +1$). Find a shortest path from
$(0,0)$ to a node $(k, \sigma)$, where $\sigma \geq \mu$. This is a "flow
problem" in an "exponentially" large graph.


\subsection{Two approximation results}

\paragraph{Theorem: Let $\frac{c_1}{a_1} \geq \frac{c_2}{a_2}\geq \dots \geq
\frac{c_n}{a_n}$. With $\mathcal{O}(n)$ comparisons we can determine $\bar{x}
\in \{0,1\}^n$, $a^T \bar{x} \leq a_0$ (feasible) and $c^T \bar{x} \geq
\frac{1}{2} \gamma (n, a_0)$. }

\subparagraph{Proof:}
Let $s$ be such that $\sum_1^s a_i \leq a_0$ but $\sum_1^s a_i + a_s + 1 > a_0$
wlog, $a_i \leq a_0$, $\forall i s \geq 1$.\\
$\gamma (n, a_O) \leq \sum_1^s c_i + c_{s+1} \frac{a_0 - \sum_1^s a_i}{a_{s+1}}
\leq \sum_1^s c_i + c_{s+1} = \underbrace{c^T x_1}_{\text{all elements until
$s$}} + \underbrace{c^T x_2}_{\text{just $s+1$ element, since
$a_0 \geq a_i$}}$.\\
Where $x_1$ and $x_2$ are feasible $0/1$-solutions.\\
$c^T x_1 + c^T x_2 \leq 2 \max \{c^T x_1, c^T x_2\}$.

\paragraph{FPTAS: Fully polynomial time approximation scheme} A family of
algorithms $\mathcal{A}_\epsilon$ such that $\mathcal{A}_\epsilon$ finds a
feasible $0/1$-solution for knapsack of value $c(\bar{x}) \geq (1-\epsilon)
\times OPT$ (max problem) or $c(\bar{x}) \leq (1+\epsilon) \times OPT$ (min
problem), where $OPT$ is the optimal objective value, in running time
polynomial in $\frac{1}{\epsilon}$ and in the binary encoding of the instance.

\paragraph{Theorem - FPTAS} Let $\epsilon > 0$. In time $\mathcal{O}(n^3
\frac{1}{\epsilon} log (\frac{n^2}{\epsilon}))$ we can compute a feasible
solution $\bar{x} \in \{0,1\}^n$, $a^T \bar{x} \leq a_0$, such that $c^T
\bar{x} \geq (1-\epsilon) \gamma (n, a_0)$ ($\bar{x}$ is not far away from the
optimal solution).

\subparagraph{Proof:}
Let $\Delta = \frac{\epsilon c_{max}}{n} \rightarrow \epsilon = \frac{\Delta n}
{c_{max}} \rightarrow c_{max} = \frac{\Delta n}{\epsilon}$.\\
$\frac{c^T}{\Delta} \geq c^\prime = (\lfloor \frac{c_1}{\Delta} \rfloor, \dots,
\lfloor \frac{c_n}{\Delta} \rfloor) \geq (\frac{c_1}{\Delta} - 1, \dots,
\frac{c_n}{\Delta} - 1)$ (*)\\
$c'_{max} \leq \frac{c_{max}}{\Delta} = \frac{n}{\epsilon}$\\
Compute an optimal solution for $\max \{c^{\prime T} x \mid a^T x \leq a_0,
x \in \{0,1\}^n \}$. This can be done with the lemma before ($\alpha, \gamma$-
relation) in time $\mathcal{O}(n^3 \frac{1}{\epsilon} log(\frac{n^2}
{\epsilon}))$. Let $\bar{x}$ be this solution. Let $x^*$ be an optimal solution 
for $\max \{c^T x \mid a^T x \leq a_0, x \in \{0,1\}^n\} \rightarrow c^T x^* =
\gamma (n, a_0)$.\\
$(1-\epsilon) \gamma (n,a_0) - \epsilon \gamma (n, a_0) = \gamma (n,a_0) -
\frac{\Delta n}{c_{max}} \underbrace{\gamma (n, a_0)}_{\geq c_{max}} \leq 
\gamma (n, a_0) - \Delta n = c^T x^* - \Delta n \leq c^T x^* - \Delta
\mathds{1}^T x^* \rightarrow \Delta (\frac{c_1}{\Delta} - 1, \dots, \frac{c_n}
{\Delta}-1) x^* \underbrace{\leq}_{\text{(*)}} \Delta c^{\prime T} x^* \leq 
\Delta c^{T} \bar{x} \leq c^T x$.

\end{document}
