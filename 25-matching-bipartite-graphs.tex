\documentclass[main]{subfiles}
\begin{document}

%@@@@@@@@@@@@@@@@@@@@@@@@@@@@@@
% Main Topics: Matching number, edge cover number, stable set number, node
% cover number, theorem: \alpha(G) + \tau(G) = |V| = \rho(G) + \nu(G),
% M-Augmenting paths, combinatorial proof of König's Theorem
% Matchings in bipartite graphs - 14.12.2017
% author: Vanessa Leite

\section{Matchings in bipartite graphs}

\paragraph{Notation: Let $G=(V,E)$ be a graph. $e \in E$ is viewed as a subset
of nodes of cardinality $2$.}

\subparagraph{$deg(v) = \abs{\{v,u\} \in E}$}
\subparagraph{$deg_F(v) = \abs{\{v,u\} \in F}$, $F \subseteq E$}

\subparagraph{A path is a sequence of nodes $(v_0, v_1, \dots, v_t)$ st $\{v_i,
v_{i+1}\} \in E$, $\forall i \in 0, \dots, t-1$}

\subparagraph{$M \subseteq E$ is a matching if $\forall e,e^\prime \in M$,
$e \neq e^\prime$, $e \cap e^\prime = \emptyset$}
A perfect matching is a matching that covers all vertices, i.e., it has size
$\frac{\abs{V}}{2}$.

\subparagraph{$F \subseteq E$ is an edge cover if $\forall v \in V$,
$v \cap e = \emptyset$, for some $e \in F$. Edge cover is the set $F$ of edges
such that every vertex of $V$ is incident to some edge in $F$. $F$ covers $V$.}

\subparagraph{$W \subseteq V$ is a node cover if $\forall e \in E$,
$e \cap w = \emptyset$, for some $w \in W$. Node or vertex cover, is the set
$W$ of vertices that contains at least one endpoint of every edge. This way,
$W$ covers $E$.}

\subparagraph{$S \subseteq V$ is a stable set if $e \nsubseteq S$ $\forall e
\in E$. Stable set is a subset of vertices $S$ such that no two vertices in $S$
are adjacent in $G$.}

\paragraph{Definition - graph parameters}

\subparagraph{$\alpha(G)$ is the maximum size of a stable set in $G$}

\subparagraph{$\rho(G) = \min \{ \abs{F} \mid F \subseteq E \text{ edge cover}
\}$}

\subparagraph{$\tau(G) = \min \{ \abs{W} \mid W \subseteq V \text{ node cover}
\}$}

\subparagraph{$v(G)$ is the maximum size of a matching in $G$}

\paragraph{Observations:}
\begin{enumerate}
\item $S \subseteq V$ stable iff $V \setminus S$ is a node cover
\item $\alpha(G) \leq \rho(G)$
\item $v(G) \leq \tau(G)$
\end{enumerate}
\subparagraph{Proof:}
Consider that G has no isolated node.
\begin{enumerate}
\item $S \subseteq V$ stable $\iff \forall e \in E$, $e \nsubseteq S \iff
\forall e \in E$, $e \cap (V \setminus S) \neq \emptyset \iff V \setminus S$
is a node cover.
\item Let $S \subseteq V$ be stable and $F \subseteq E$ an edge cover. In
particular, $F$ must cover all nodes in $S$, thus, $\alpha(G) \leq \rho(G)$.
\item Let $M \subseteq E$ be a matching and $W \subseteq V$ is a node cover. In
particular, $W$ must cover all edges on $M$, thus, $v(G) \leq \tau(G)$.
\end{enumerate}

\paragraph{Theorem: For any graph $G=(V,E)$ without isolated nodes, $\alpha(G)
+ \tau(G) = \abs{V} = v(G) + \rho(G)$}

\subparagraph{Proof:}
$\alpha(G) + \tau(G) = \abs{V}$ follows from observation.\\
Let $M$ be a maximum matching, $\abs{M} = v(G)$. $M$ misses $\abs{V} -
2\abs{M}$ nodes. For every "missed" node, pick an edge incident to it. This
gives $T \subseteq E$. $M \cup T$ is an edge cover of size $\abs{M} + \abs{V}
- 2\abs{M} = \abs{V} - \abs{M} \rightarrow \rho(G) \leq \abs{V} - \abs{M}
\rightarrow \rho(G) + \underbrace{\abs{M}}_{=v(G)} \leq \abs{V}$.\\
Conversely, let $F \subseteq E$ be an edge cover, $\abs{F} = \rho(G)$. Note
that if $\{u,v\} \in F$ then $deg_F(u) \leq 1$ or $deg_F(v) \leq 1$. For every
$v \in V$ st $deg_F(v) > 1$ remove arbitrarily $deg_F(v) -1$ many edges from
$F$. This operation leads to a matching of size $\abs{F} - \sum_{v \in V}
(deg_F(v) -1) = \abs{F} + \underbrace{\abs{V}}_{\text{count each node}} -
\underbrace{2\abs{F}}_{count each edge twice} = -\abs{F} + \abs{V} \rightarrow
v(G) \geq \abs{V} - \rho(G) \rightarrow v(G) + \rho(G) \geq \abs{V}$.
Thus, $\rho(G) + v(G) = \abs{V}$.

\paragraph{Definition - $M$-augmenting path}
$M \subseteq E$ is a matching. A path $(v_0, \dots, v_t)$ is $M$-augmenting if:
\begin{itemize}
\item $t$ is odd and all nodes $v_i, i = 1, \dots, t$ are distincts.
\item $\{v_1, v_2\}, \dots, \{v_{t-2}, v_{t-1}\} \in M$.
\item $v_0$ and $v_t$ are nodes missed by $M$.
\end{itemize}

\emph{Note: if $(v_0, \dots, v_t)$ is $M$-augmenting path, then $\{v_0, v_1\},
\{v2, v_3\}, \dots, \{v_{t-1}, v_t\} \notin M$ and $\bigcup_{i=0}^{t-1} \{v_i,
v_{i+1}\} = M \cup M^\prime$, where $M^\prime$ is a matching of size $\abs{M}
+ 1$.}

\paragraph{Remark: Let $M$ be a matching, $P$ an $M$-augmenting path. Then,
$M\Delta E[P]$ gives a larger matching, where $A\Delta B = (A \cup B) \setminus
(A \cap B) = (A\setminus B) \cup (B \setminus A)$ is the symmetric difference
of the sets $A$ and $B$.}

\paragraph{Theorem: For any graph $G=(V,E)$ a current matching $M \subseteq E$
is either of max size ($\abs{M} = v(G)$) or there exists an $M$-augmenting
path.}
\subparagraph{Proof:}
If $\abs{M} = v(G)$ then, no augmenting path exists.
Suppose $\abs{M} < v(G)$. Let $M^\prime$ be a matching, $\abs{M^\prime} >
\abs{M}$. Consider subgraph $(V, M \cup M^\prime)$. $G$ decomposis in connected
components: paths, cycles and/or isolated components. As $\abs{M^\prime} >
\abs{M}$ there exists a connected component in $(V, M \cup M^\prime)$ that is
an augmenting path.

\paragraph{Algorithm for computing $v(G)$ when $G$ is bipartite}
Note: $G$ bipartite means $V = U \cup W$ and $\forall e in E$, $e = (u,w)$,
where $u \in U$, $w \in W$. It also implies that $U$ is a stable set and $W$ is
also a stable set.

\begin{enumerate}
\item $M = \emptyset$
\item Find an $M$-augmenting path and update $M$ as long as possible.
\end{enumerate}

\subparagraph{How to find $M$-augmenting path when $G$ is bipartite?}
Pick $u \in U$ missed by $M$. Let $w \in W$ be mised by $M$. Turn all edges in
$M$ from $W$ to $U$ and all edges in $E \setminus M$ from $U$ to $W$. Test
wether there exists a directed path from $u$ to $w$ in the digraph $D=(V,A)$
with arc set $A$ defined before. Using BFS, this task can be accomplished in
time $\mathcal{O}(\abs{E})$.

\paragraph{Theorem: $v(G)$ can be determined, when $G$ is bipartite, in time
$\mathcal{O}(\abs{V} \abs{E})$}

\paragraph{Konig's Theorem: In a bipartite graph $G$: $v(G) = \tau(G)$}
\subparagraph{Proof:}
From observation, show $v(G) = \tau(G)$. Perform induction on $\abs{V}$\\
For $G$ contains one edge, the number of vertices is $\abs{V} = 2$, results
trivial.
\textbf{Claim: $\exists u \in V$ that is covered by all max size matchings.}\\
Consider $G^\prime = G \setminus u$. From hypothesis of induction $v(G^\prime)
= \tau(G^\prime) \rightarrow v(G) = v(G^\prime) + 1$.
Let $W^\prime$ be a node cover in $G^\prime$ of size $\tau(G^\prime)$.
$W^\prime \cup \{u\}$ is a node cover such that $\abs{W^\prime \cup \{u\}} =
\tau(G^\prime) + 1 = v(G)$.

\subparagraph{Proof of the claim:}
Suppose not true. For every edge $e = \{u,v\} \in E$, there exists a maximum
size matching not covering $u$ or $v$. So, there exists a maximum matching $M$
covering $v$ but not $u$ and there exists a maximum matchin $N$ covering $u$
but not $v$. Consider $G^\prime = \{V, M \cup U\}$. The component of $G^\prime$
containing $u$ is a path $P$. If $\abs{E(P)}$ is odd, $P$ is $M$-augmenting
path, which is a contradiction. Otherwise, if $\abs{E(P)}$ is even, $E(P) \cup
\{e\}$ is $N$-augmenting path, contradiction.

\paragraph{Konig's edge cover Theorem: For any bipartite graph without isolated
vertices, $\alpha(G) = \rho(G)$.}
Isolated vertice: ?
\end{document}