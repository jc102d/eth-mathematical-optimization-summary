\documentclass[main]{subfiles}
\begin{document}

%@@@@@@@@@@@@@@@@@@@@@@@@@@@@@@
% Main Topics: Independence systems, the rank function, submodularity and the
% connection to matroids, the greedy algorithm
% Independence Systems and Matroids - 07.12.2017
% author: Vanessa Leite

\section{Independence Systems and Matroids}

\paragraph{Definition - Independece system}
Let $N$ be a finite set, $N = \{1, \dots, n\}$. $\mathcal{I}$ is a collection
of subsets of $N$. $(N, \mathcal{I})$ is an independence system if it has the
properties:
\begin{itemize}
\itemsep0em
\item $\emptyset \in \mathcal{I}$
\item $A \in \mathcal{I}$ and $B \subseteq A \rightarrow B \in \mathcal{I}$
\end{itemize}

Usually, $\mathcal{I}$ is implicitly defined, for example, look at all vertices
in a graph that do not form a cycle.\\

\paragraph{Matroid - definition}
A matroid $M$ is a finite collection of finite sets that satisfies three
axioms:\\
1. Non-emptiness: the empty set is in $M$. (Thus, $M$ is not itself
empty.)\\
2. Heredity: If a set $X$ is an element of $M$, then every subset of $X$ is
also in $M$.\\
3. Exchange: If $X$ and $Y$ are two sets in $M$ where $|X|>|Y|$, then there is
an element $x \in X \setminus Y$ such that $Y \cup \{x\} $ is in $M$.

Examples:
\begin{enumerate}
\itemsep0em
\item Linear independence: $A \in \Z^{m \times n}$, $\mathcal{I}$ is a
collection of sets in $\{1, \dots, n\}$ st $\{A_i \mid i \in \mathcal{I}\}$ is
linearly independent whenever $i \in \mathcal{I}$.
\item $G=(V,E)$ is given, $I \subseteq E$ is independent if $(V,I)$ does not
contain a cycle and $\mathcal{I} = \{ I: \text{ does not contain a cycle}\}$.
\item Knapsack problem: $\{x \in \{0,1\}^n \mid a_1 x_1 + \dots + a_n x_n \leq
b \}$, $a_i \in \Z_+\setminus \{0\}$ and $b \in \Z_+ \setminus \{0\}$. $N = \{
1, \dots, n\}$. $I \subseteq N$ is independent if $\sum_{i \in I} a_i \leq b$.
$\mathcal{I} = \{I \subseteq N \mid \sum_{i \in I} a_i \leq b\}$ is an
independent system. This is not true if $A_{\cdot i} \ngeq 0$.
\end{enumerate}

\paragraph{Definition: Let $(N,\mathcal{I})$ be an independent system. Then
each member $I \in \mathcal{I}$ is independent. $I \notin \mathcal{I}$ is
dependent.}
Let $S \subseteq N$, $F \subseteq S$, $F \in \mathcal{I}$. $F$ is a basis of
$S$ if we can't enlarge it: $F \cup \{i\} \notin \mathcal{I}, \forall i \in
S\setminus \{F\}$. A basis of $S$ is not necessarily a basis of $N$.

The rank of $S$ is $r(S) = \max \{|F|: F \text{ basis of } S\}$.

$I \notin \mathcal{I}$, $I \subseteq N$ is a circuit if $I\setminus \{i\} \in
\mathcal{I}, \forall i \in I$.
Unique circuit property: Let $M=(E,I)$ be a matroid. Let $S\in I$ and $e$ st
$S+e \notin I$.  Then there exists a unique circuit $C\subseteq S+e$. Proof:
Suppose $S+e$ contains more than one circuit, say $C1$ and$C2$ with $C1\neq
C2$. By minimality of $C1$ and $C2$, we have that there exists $f\in C1
\setminus C2$ and $g \in C2\setminus C1$.  Since $C1-f \in I$ (by minimality
of the circuit $C1$), we can extend it to a maximal independent set $X$ of
$S+e$. Since $S$ is also independent, we must have that $|X|=|S|$ and since 
$e \in C1-f$, we must have that $X=S+e-f \in I$. But this means that 
$C2\subseteq S+e-f=X$ which is a contradiction since $C2$ is dependent.


A independence system $(N,\mathcal{I})$ is a matroid if $\forall S \subseteq
N$, every basis of $S$ has the same cardinality.\\
Example: Consider $\{x \in \{0,1\}^3 \mid x_1 + x_2 + 2x_3 \leq 2\}$.\\
$S = \{1, 2, 3\} = N$. $\{1,2\}$ and $\{3\}$ are basis of $S$. Both are
independent and cannot be enlarged, but they do not have the same cardinality,
this way, knanpsack problems are not matroids in general.

\paragraph{Definition: Given a finite set $N$. A function of a set (set
function) $f: 2^N \mapsto \R$ is submodular if $\forall S$, $T \subseteq N$,
then $f(S) + f(T) \geq f(S \cup T) + f(S \cap T)$. (supermodular $\leq$ instead
of $\geq$)}

\subparagraph{$f$ non decreasing if $f(S) \leq f(T) \forall S \subseteq T$.}

\paragraph{Theorem:}
\begin{enumerate}
\itemsep0em
\item The set function $f$ is submodular iff $\forall j, k$, $j \neq k$ and
$A \subseteq N \setminus \{j,k\}$ then $f(A \cup \{j\}) - f(A) \geq f(A \cup
\{j,k\}) - f(A \cup \{k\})$ (diminishing return property)
\item $f$ is submodular and nondecreasing iff $\forall S, T \subseteq N$:
$f(T) \leq f(S) + \sum_{j \in T\setminus S} [f(S \cup \{j\}) - f(S)]$.
\subitem $r(S) + r(T\setminus S) \geq |A \cap S| + |A \cap (T\setminus S)| =
|A| = r(T)$
\end{enumerate}

\paragraph{$(N,\mathcal{I})$ is a matroid if, for all $F \subseteq N$, every
maximal independent set contained in $F$ has the same cardinality $r(F)$.}

\paragraph{Theorem: $(N,\mathcal{I})$ is a matroid iff its rank function is
submodular.}

\subparagraph{Proof:}
\begin{itemize}
\itemsep0em
\item "$\rightarrow$": $(N,\mathcal{I})$ ind. system then rank function 
nondecreasing.
\subitem $r(S \cup \{j\}) - r(S) \leq 1$, $r(\emptyset) = 0$. Let
$(N,\mathcal{I})$ be a matroid: $r$ satisfies diminishing return property in
previous theorem. We want to show that $r(S \cup \{j\}) - r(S) \geq  r(S \cup
\{j,k\}) - r(S \cup \{k\})$.\\
This is satisfied whenever $r(S \cup \{j\}) - r(S) = 1$ or $r(S \cup \{j,k\}
- r(S \cup \{k\}) = 0$.\\
Suppose $r(S \cup \{j\}) = r(S) = a$. $r(S \cup \{j,k\}) - r(S \cup \{k\}) = 1
\rightarrow r(S \cup \{j,k\}) = a+ 1 \rightarrow a = r(S \cup \{k\})$.\\
Let $Q$ be a basis of $S$ st $|Q| = a$. $Q \cup \{i\} \notin \mathcal{I}$,
$\forall i \in S\setminus \{Q\}$.\\
$Q \cup \{j\} \notin \mathcal{I} \rightarrow Q$ basis of $S \cup \{j\}$.\\
$Q \cup \{k\} \notin \mathcal{I} \rightarrow Q$ basis of $S \cup \{k\}$.\\
$Q$ is a basis of $S \cup \{j,k\}$, which is a contradiction because $|Q| = a$,
$r(S \cup \{j,k\}) = a+1$.
\item "$\leftarrow$": assume $r$ is submodular and nondecreasing.
\subitem Suppose $(N,\mathcal{I})$ is not a matroid. There exists $T \subseteq
N$ and basis $S_1$ and $S_2$ of $T$ st $|S_1| < |S_2|$.\\
Check (2) from previous theorem with $T = S_2$ and $S = S_1$:\\
$r(S_2) \leq r(S_1) + \sum_{j \in S_2 \setminus S_1}( r(S_1 \cup \{j\}) -
r(S_1)) \rightarrow \exists j \in S_2 \setminus S_1$ st $r(S_1 \cup \{j\}) =
r(S_1) + 1$\\
$S_1 \cup \{j\} \subseteq T$ and $S_1 \cup \{j\} \in \mathcal{I}$, and this is
a contradiction because $S_1$ was basis of $T$ and thus, should have maximum
cardinality.
\end{itemize}

\paragraph{Summary}
The sets in a matroid $M$ are typically called independent sets; for example,
we would say that any subset of an independent set is independent. The union of
all sets in $M$ is called the ground set. An independent set is called a basis
if it is not a proper subset of another independent set. The exchange property
implies that every basis of a matroid has the same cardinality. The rank of
a subset $X$ of the ground set is the size of the largest independent subset of 
$X$. A subset of theground set that is not in $M$ is called dependent. A
dependent set is called a circuit if every proper subset is independent.
\subsection{Optimization over matroids}

Consider $\max \sum_{j=1}^n c_j x_j$. $(N,\mathcal{I})$ independence system,
$c_j \in \Z_+$.
$\sum_{j \in S} x_j \leq r(S)$, $\forall S \subseteq N$, $x_j \in \{0,1\}
\forall j$ (to find maximum independence set)

Relaxation problem: Let $P_r = \{x \in [0,1]^n \mid \sum_{j \in S} x_j \leq
r(S), \forall S \subseteq N\}$. $F = P_r \cap \Z^n$.

\paragraph{Theorem: Let $r$ be a submodular function $r(S) \in \Z_+$, $\forall
S \subseteq N$, $r$ non decreasing, $r(\emptyset) = 0$. $P_r = conv(P_r \cap
\Z^n)$, i.e., all the extreme points are integers.}

\subparagraph{Proof:}
Consider $\max \sum c_j x_j$ st $x \in P_r$, and its dual is $\min \{\sum_{S
\subseteq N} r(S)y_s \mid \sum_{s: j \in S} y_s \geq c_j, y_s \geq 0, \forall
S \subseteq N, c \in \Z^n\}$. Wlog: $c_1 \geq c_2 \geq \dots \geq c_k > 0 \geq
c_{k+1} \geq \dots \geq c_n$ (the values are sorted).\\
Define $S^0 = \emptyset$, $S^j = \{1, \dots, j\}$ and
\begin{itemize}
\itemsep0em
\item $x_j = r(S^j) - r(S^{j-1})$ for $j \leq k-1$ and $x_j = 0$, elsewhere.
\item $y_s = 0 \forall S \neq S^1, \dots, S^k$ and $y_{s^k} = c_k$, $y_{s^j} =
c_j - c_{j+1} \forall 1 \leq j \leq k-1$.
\end{itemize}
$x,y$ are integral vectors and $x,y$ are both non negatives vectors.\\
$\sum_{j \in T} x_j = \sum_{j \in T, j \leq k}(r(S^j) - r(S^{j-1}))
\underbrace{\leq}_{\text{definition of submodularity}} \sum_{j \in T, j \leq k}
(r(S^j \cap T) - r(S^{j-1} \cap T) \leq \sum_{j=1}^k (r(S^j \cap T) - r(S^{j-1}
\cap T)) = r(S^k \cap T) - r(\emptyset) \leq r(T)$.\\
In the dual, $\sum_{s: j \in S} y_s = y_{s^j} + \dots + y_{s^k} = c_j$ $\forall
j \leq k$. $\sum_{s: j \in S, j > k} y_s = 0 > c_j$.\\
primal value equals to dual value: $\sum_{j=1}^k c_j(r(S^j) - r(S^{j-1})) =
\sum_{j=1}^{k-1}\underbrace{(c_j - c_{j+1})}_{y_{s^j}} r(S^j) +
\underbrace{c_k r(S^k)}_{y_{s^k}}$.

\end{document}