\documentclass[main]{subfiles}
\begin{document}

%@@@@@@@@@@@@@@@@@@@@@@@@@@@@@@
% Main Topics: consequences of projections, Farkas Lemma, standard form
% polyhedra, cones
% author: Vanessa Leite

\section{Farkas Lemma and Standard Form Polyhedra}

\subsection{Some consequences from lecture 4}

\paragraph{Corollary: If I take a polyhedron $P = \{ x \in \mathbb{R}^n \mid Ax
\leq b\}$, the projection $proj_{(x_1, \dots, x_i}(P)$ is a polyhedron (by
removing points). }

\paragraph{Corollary: Take a polyhedron $P = \{ x \in \mathbb{R}^n \mid Ax \leq
b\}$, consider the set $Q$ (the set of all vectors of the form: $\{ wx \mid x
\in P \}$, where $w \in \mathbb{Q}^{d \times n}$; $wx \subseteq \mathbb{R}^d$.
$Q$ is a polyhedron.}

\subparagraph{Proof:}
Let's define another set $D = \{(x,y): wx - y = 0$, $x \in P\} = \{(x,y) \mid wx -y = 0$, $Ax \leq b\} \subseteq \mathbb{R}^{n+d}$. $Q$ is the projection of $D$ on the space of $Q = proj_y (D) = \{ y \in \mathbb{R}^d \mid \exists x \in \mathbb{R}^{d \times n}$ such that $\underbrace{(x,y) \in D}_{\text{that means $wx =y$}} \}$.

\subsection{Standard form polyhedra}

\end{document}